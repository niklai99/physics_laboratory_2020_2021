%----------------------------------------------------------------------------------------
%	PACKAGES AND OTHER DOCUMENT CONFIGURATIONS
%----------------------------------------------------------------------------------------

\documentclass[a4paper,11pt]{article} % Font and paper size

%----------------------------------------------------------------------------------------
%	PACKAGES AND OTHER DOCUMENT CONFIGURATIONS
%----------------------------------------------------------------------------------------

\usepackage[utf8]{inputenc} % Required for inputting international characters
\usepackage[T1]{fontenc} % Output font encoding for international characters
\usepackage[italian]{babel} % Italian dictionary


\usepackage[table]{xcolor} % Required for custom colors
\usepackage{gensymb}
\usepackage{amsmath}
\usepackage{bm}
\usepackage{tikz}
\usepackage{hhline}
\usepackage{listings}
\usepackage{enumitem}

%\usepackage[margin=2cm, includefoot]{geometry} % Modify margins
\usepackage{fancyhdr}
\usepackage{rotating}
\usepackage[hidelinks]{hyperref} % Hyperlinks

\usepackage[none]{hyphenat}% Non spezza le parole nelle tabelle
\usepackage{array}

\usepackage{graphicx} % Required for figures
\usepackage{float}
\usepackage{wrapfig}
\usepackage{caption}
\usepackage{subcaption}

%pagestyle
\pagestyle{fancy}
\fancyhead{}
\fancyfoot{}
\fancyfoot[R]{\thepage}
\renewcommand{\headrulewidth}{0pt}



\usepackage{xfrac}
\usepackage{amssymb}

\usepackage{multicol}
\usepackage{multirow}

\usepackage[toc, page]{appendix}
\usepackage{booktabs}
\usepackage{siunitx}


%----------------------------------------------------------------------------------------
%	NEW COMMANDS
%----------------------------------------------------------------------------------------

\newcommand{\restr}[2]{{% we make the whole thing an ordinary symbol
\left.\kern-\nulldelimiterspace % automatically resize the bar with \right
#1 % the function

\right|_{#2} % this is the delimiter
}}

\newcommand{\tnhl}{\tabularnewline\hline}
\newcommand{\tn}{\tabularnewline}
\newcolumntype{x}[1]{%
	>{\centering\hspace{0pt}}p{#1}}%


 % Include the file specifying document layout and packages


%----------------------------------------------------------------------------------------
%	GENERAL INFORMATION 
%----------------------------------------------------------------------------------------

\newcommand{\labcourse}{Laboratorio di Fisica}
\newcommand{\teacher}{Docenti: Prof. A. Garfagnini - Prof. M. Lunardon}
\newcommand{\laurea}{Corso di Laurea in Fisica}
\newcommand{\channel}{Canale 1 A-L}
\newcommand{\academicyear}{Anno Accademico 2020/2021}
\newcommand{\labexp}{Esperienza di Laboratorio}
\newcommand{\exptitle}{Amplificatori Operazionali \& Calibrazione Arduino}
\newcommand{\turno}{Turno T2}
\newcommand{\name}{Nicolò Lai}
\newcommand{\matricola}{1193976}
\newcommand{\mail}{nicolo.lai@studenti.unipd.it}
\newcommand{\consegna}{Data Esperienza}
\newcommand{\data}{28/10/2020 - 29/10/2020}


%----------------------------------------------------------------------------------------
%	DOCUMENT 
%----------------------------------------------------------------------------------------

\begin{document}
\def\sectionautorefname{Sezione} 
\def\subsectionautorefname{Sezione} 
\def\subsubsectionautorefname{Sezione}

%----------------------------------------------------------------------------------------
%	TITLE PAGE
%----------------------------------------------------------------------------------------

	\begin{titlepage}

		\begin{center}
			\Huge{\bfseries \labcourse}\\
				
			\LARGE \teacher \\
			\Large \laurea\\
			\Large \channel\\
			\Large \academicyear\\
			[1cm] 
			\line(1,0){400}\\
			[3.5cm]
				
			\textsc{\huge{\bfseries \labexp}}\\
			\huge{\exptitle}\\
			[2mm] \line(1,0){300}\\
			[10cm]
		\end{center}
		
		
		\begin{flushleft}
			\textsc{\Large \turno}\\
			[0.5cm] \textsc{\large {\bfseries \name}} \\ 
			\indent\large \matricola \\ 
			\indent\large \mail \\
		\end{flushleft}
			
						
		\begin{flushright}
				\textsc{\Large\consegna}\\
				\textsc{\large \data}					
		\end{flushright}
				
	\end{titlepage}
\cleardoublepage


%----------------------------------------------------------------------------------------
%	APPARATO SPERIMENTALE
%----------------------------------------------------------------------------------------

\section{Obiettivo}
Verificare la linearità di un amplificatore operazionale e misurare l'amplificazione di un circuito che lo comprenda.
Misurare la frequenza di taglio di un filtro attivo. Calcolare il sampling rate e la funzione di calibrazione in
tensione di una scheda Arduino Due.

%----------------------------------------------------------------------------------------
%	APPARATO SPERIMENTALE
%----------------------------------------------------------------------------------------

%\section{Strumentazione e Componenti}\label{s:strumenti}
%
%\begin{itemize}
%	\item \textbf{Oscilloscopio} (Tektronix TBS1102B): Lo strumento presenta un'accuratezza sul guadagno verticale $\Delta_{g}$ pari
%	al 3\% del valore letto (errore massimo) ed è \textit{generalmente} il contributo più significativo. L'incertezza di
%	guadagno sui tempi si assume trascurabile. L'accuratezza che tiene conto degli effetti di risoluzione e imprecisione
%	della traccia $\Delta_{l}$ è di 1/10 di divisione su tutta la scala di lettura (errore massimo), uguale sia per le tensioni sia
%	per i tempi.
%
%	\item \textbf{Generatore di funzioni} (Tektronix AFG1022)
%
%	\item  \textbf{Alimentatore di tensione continua}: Lo strumento presenta due uscite con erogazione di tensione tra 0
%	e 20 \si{\volt} e un'uscita con erogazione fissata a 5 \si{\volt}
%	
%	\item \textbf{Multimetro digitale} (Metrix MTX3292): Si riporta l'accuratezza dello strumento, per misure di
%	resistenza e di capacità, relativa unicamente ai fondoscala utilizzati nell'esperienza.
%
%	\begin{table}[H]
%		\small
%		\centering
%		\begin{tabular}{x{2cm} x{3cm} x{3cm} } \toprule[0.5px]\toprule[0.1px]
%			
%			\multicolumn{3}{c}{Accuratezza Metrix MTX3292}\tn
%			\midrule[0.1px]
%			
%			F.S. & Precisione & Risoluzione \tn
%			
%			\addlinespace
%			
%			1   \si{k\ohm} & 0.10\% + 8  & 0.01 \si{\ohm}  \tn 10  \si{k\ohm} & 0.07\% + 8  & 0.1  \si{\ohm}  \tn 100
%			\si{k\ohm} & 0.07\% + 8  & 1 \si{\ohm}  \tn
%
%			
%			\addlinespace
%
%			1000 \si{p\farad}         & 2.5\% + 15  & 1 \si{p\farad}   \tn
%			
%			\bottomrule[0.5px]
%			
%			
%		\end{tabular}
%		\caption{Per i fondoscala indicati si riportano la precisione (contributo di scala in percentuale e contributo
%		di lettura sul digit meno significativo) e la risoluzione dello strumento.}
%		\label{t:metrix}
%	\end{table}	 
%
%	\item \textbf{Componenti circuitali} (Resistori e Condensatori): Si riportano i valori delle resistenze e capacità
%	utilizzate per l'assemblamento dei circuiti utilizzati nel corso dell'esperienza, misurate preliminarmente con il
%	multimetro digitale Metrix. 
%
%	\begin{table}[H]
%		\small
%		\centering
%		\begin{tabular}{x{2cm} x{3cm} x{3cm} } \toprule[0.5px]\toprule[0.1px]
%			
%			\multicolumn{3}{c}{Resistori e Condensatori}\tn
%			\midrule[0.1px]
%			
%			Resistenza & Valore & F.S. \tn
%			
%			\addlinespace
%			
%			$R_{\text{f}}$ & $(82.46 \pm 0.03)\,\si{k\ohm}$ & $100\,\si{k\ohm}$ \tn
%
%			$R_1$ & $(8.089 \pm 0.003)\,\si{k\ohm}$ & $10\,\si{k\ohm}$ \tn
%
%			$R_3$ & $(46.54 \pm 0.05)\,\si{\ohm}$ & $1\,\si{k\ohm}$ \tn
%		
%			\addlinespace
%
%			\midrule[0.1px]
%			
%			Capacità & Valore & F.S. \tn
%			
%			\addlinespace
%
%			$C_1$  & $(977 \pm 17)\,\si{p\farad}$  & $1000\,\si{p\farad}$   \tn
%			
%			\bottomrule[0.5px]
%			
%		\end{tabular}
%		\caption{In tabella si indicano le componenti circuitali (resistori e capacità) utilizzando delle label
%		specifiche per ciascuna di esse: questa notazione è costante nel corso dell'esperienza.}
%		\label{t:direct_measures}
%	\end{table}	
%
%	\item \textbf{Circuito integrato TL082C} (Due Amplificatori Operazionali): essendo gli amplificatori operazionali
%	delle componenti circuitali attive, esse devono essere alimentate. Si utilizza dunque il generatore di tensione
%	continua con $V_{\text{cc}} = +15\,\si{\volt}$ e $V_{\text{ee}}=-15\,\si{\volt}$ per l'alimentazione
%	dell'amplificatore operazionale utilizzato nell'esperienza. Nel corso di quest'ultima, si assume un comportamento
%	\textit{ideale} dell'amplificatore operazionale, ovvero che il polo positivo ed il polo negativo si trovino
%	allo stesso potenziale.
%	
%	\item \textbf{Scheda Arduino Due}
%\end{itemize}
%
%\cleardoublepage


%----------------------------------------------------------------------------------------
%	AMPLIFICATORE OPERAZIONALE INVERTENTE
%----------------------------------------------------------------------------------------

\section{Amplificatore Operazionale Invertente}
In questa sezione ci si propone di studiare il comportamento di un circuito puramente resistivo comprendente un
amplificatore operazionale in configurazione invertente (polo positivo a massa, polo negativo collegato al segnale in
ingresso). Si vuole in particolare verificare la sua linearità e stimare l'amplificazione del circuito come grandezza
derivata sia partendo dalle misure dirette delle resistenze sia come parametro di un'interpolazione lineare di misure
acquisite con l'oscilloscopio. 


%----------------------------------------------------------------------------------------
%	CONFIGURAZIONE SPERIMENTALE
%----------------------------------------------------------------------------------------

\subsection{Configurazione Sperimentale}\label{s:guadagno}

Si inizia assemblando il circuito, rappresentato in \autoref{i:opamp_circuit}, utilizzando le resistenze $R_{\text{f}}$, $R_1$,
$R_3$ e l'amplificatore operazionale. La resistenza $R_{\text{g}}$ rappresenta la resistenza interna del generatore, non nulla in
quanto ci si trova in condizioni di non idealità. Le resistenze esterne, invece, vengono misurate

\begin{wrapfigure}{R}{0.6\textwidth}
	\centering
	\includegraphics[width=0.6\textwidth]{../Simulations/OpAmp/circuit_image_nosim.png}
	\caption{\footnotesize Rappresentazione a variabili concentrate del circuito assemblato in laboratorio.}
	\label{i:opamp_circuit}
\end{wrapfigure}

\noindent direttamente utilizzando il multimetro Metrix MTX3292, ottenendo i risultati esposti in
\autoref{t:direct_measures}. Si utilizza poi un generatore di tensione continua con $V_{\text{cc}}=+15\,\si{\volt}$ e
$V_{\text{ee}}=-15\,\si{\volt}$ per l'alimentazione dell'amplificatore operazionale. Si assume, inoltre, che esso abbia
un comportamento ideale, ovvero che il polo positivo ed il polo negativo si trovino allo stesso potenziale. Il segnale
viene prelevato nei punti \textit{IN} e \textit{OUT} evidenziati nello schema in \autoref{i:opamp_circuit} (e verrà in
seguito richiamato rispettivamente come $V_{\text{in}}$ e $V_{\text{out}}$) utilizzando due sonde con fattore di
attenuazione 10X. Nel canale CH1 dell'oscilloscopio viene visualizzato il segnale in ingresso $V_{\text{in}}$, mentre il
segnale in uscita $V_{\text{out}}$ è prelevato dalla sonda collegata al canale CH2. 

\begin{wraptable}{L}{0.5\textwidth}
	\small
	\centering
	\begin{tabular}{x{1.8cm} x{2.5cm} x{2cm} } \toprule[0.5px]\toprule[0.1px]
		
		\multicolumn{3}{c}{Misure Dirette delle Resistenze}\tn
		\midrule[0.1px]
		
		Resistenza & Valore & F.S. \tn
		
		\addlinespace
		
		$R_{\text{f}}$ & $82.46 \pm 0.03\,\si{k\ohm}$ & $100\,\si{k\ohm}$ \tn

		$R_1$ & $8.089 \pm 0.003\,\si{k\ohm}$ & $10\,\si{k\ohm}$ \tn

		$R_3$ & $46.54 \pm 0.05\,\si{\ohm}$ & $1\,\si{k\ohm}$ \tn
		
		\bottomrule[0.5px]		
	\end{tabular}
	\caption{\footnotesize Valori di resistenza, misurati direttamente con il multimetro, e relativo fondoscala.}
	\label{t:direct_measures}
\end{wraptable}	

\noindent Per entrambi i canali viene selezionata la modalità "attenuazione sonda 10X", in modo da compensare la
riduzione del segnale dovuta alle sonde e visualizzare quindi nel display il segnale reale. Il generatore di funzioni
viene poi configurato in modalità "50 Ohm", in modo che l'impedenza d'uscita del generatore sia comparabile con
$R_3\approx 50\,\si{\ohm}$. Ci si aspetta così di trovare una tensione in ingresso $V_{\text{in}}$ in accordo con la
tensione nominale erogata dal generatore. Si imposta infine il generatore di funzioni in modo da erogare un segnale di
tipo sinusoidale con frequenza $f_{\text{gen}}=1\,\si{k\hertz}$, mantenuta costante in questa sezione, e di ampiezza
invece variabile tra $200\,\si{\mV}$ picco picco e $3.5\,\si{\V}$ picco picco. \\

\noindent Facendo riferimento ai valori delle resistenze $R_{\text{f}}$ ed $R_1$ riportate in \autoref{t:direct_measures}, si
vuole calcolare il guadagno $G$ del circuito atteso utilizzando l'\autoref{e:guadagno}, ottenuta risolvendo il circuito.
Si vuole far notare che, avendo posto l'amplificatore operazionale in configurazione invertente, si ritrova un segno
meno ad indicare un segnale in uscita $V_{\text{out}}$ "invertito" (ovvero ci si aspetta che i massimi del segnale in
ingresso corrispondano a minimi del segnale in uscita e viceversa). Tuttavia, nell'analisi che segue, verrà sempre
considerato il valore assoluto dell'amplificazione $G$, che si definisce dunque come
\begin{align}\label{e:guadagno}
	G&=\left|-\frac{R_{\text{f}}}{R_{1}}\right| = 10.194
	&
	\sigma_{G}&=\sqrt{	\left(	\frac{	1	}{	R_{1}	}	\right)^2	\sigma_{R_{\text{f}}}^2	
	+	\left(	\frac{	R_{\text{f}}	}{	R_{1}^2	}	\right)^2\sigma_{R_{1}}^2	}	= 0.006
\end{align}


%----------------------------------------------------------------------------------------
%	ACQUISIZIONE MISURE
%----------------------------------------------------------------------------------------

\subsection{Acquisizione Misure}
Al fine di verificare la linearità dell'amplificatore operazionale e stimare l'amplificazione del circuito, si decide di
far variare la tensione in ingresso impostando valori crescenti di ampiezza del segnale erogato dal generatore di
funzioni, partendo da $200\,\si{\mV}$ picco picco fino a $3.5\,\si{\V}$ picco picco. Per ciascuno di questi valori di
tensione si acquisisce la misura di un massimo e di un minimo sia del segnale $V_{\text{in}}$ sia del segnale
$V_{\text{out}}$ sfruttando i cursori orizzontali dell'oscilloscopio. In questo modo, si ottiene un campione di coppie
($V_{\text{in}}$, $V_{\text{out}}$) che ci si aspetta segua un andamento lineare, in quanto legate da
$V_{\text{out}}=-\frac{R_{\text{f}}}{R_{1}}V_{\text{in}}$ dove il segno meno, si ricorda, è dovuto all'operazionale
posto in configurazione invertente. Le misure in questione sono rappresentate in \autoref{i:opamp_eda}.


%----------------------------------------------------------------------------------------
%	SIMULAZIONE SPICE PRELIMINARE
%----------------------------------------------------------------------------------------

\subsection{Simulazione Spice Preliminare}\label{s:spice} Prima di procedere con l'analisi dati, si decide di simulare
la risposta del circuito utilizzando il programma LTSpice. Si sceglie di effettuare la simulazione considerando due
ampiezze in ingresso significative: per la prima si imposta dal generatore un'ampiezza $V_{\text{gen}}=1\,\si{\volt}$
mentre per la seconda $V_{\text{gen}}=4\,\si{\volt}$. Questa scelta è dettata dal fatto che l'amplificatore
operazionale, essendo una componente attiva del circuito, non può dare in output una tensione maggiore di quanta ne
riceve in alimentazione per conservazione dell'energia: ci si aspetta allora di trovare una situazione di saturazione
del segnale in uscita e che questa inizi a manifestarsi attorno ad un valore nominale di tensione
$V\text{pp}_{\text{gen}}=3\,\si{\volt}$ in quanto, avendo un guadagno di circa 10, il segnale in uscita non può superare
i $30\,\si{\volt}$ picco picco.

\begin{wrapfigure}{L}{0.5\textwidth}
	\centering
	\includegraphics[width=0.5\textwidth]{../Plots/Report_Plots/opamp_spice_1V_4V_BIG.png}
	\caption{\footnotesize Simulazione Spice della risposta del circuito.}
	\label{i:opamp_simulation}
\end{wrapfigure}

\noindent Dal grafico si evince chiaramente come erogando $V_{\text{gen}}=1\,\si{\volt}$ il segnale viene amplificato
correttamente di circa un fattore 10 mantenendo la forma sinusoidale, mentre erogando $V_{\text{gen}}=4\,\si{\volt}$ il
segnale in uscita presenta i picchi tagliati esattamente a livello $V_{\text{sat}}=\pm 15\,\si{\volt}$, ovvero le
tensioni di alimentazione fornite all'operazionale. Inoltre, come anticipato in \autoref{s:guadagno}, si può osservare
il comportamento invertente dell'amplificatore operazionale: ad un massimo di $V_{\text{in}}$ corrisponde un minimo di
$V_{\text{out}}$ e viceversa.\\


%----------------------------------------------------------------------------------------
%	DATI E ANALISI
%----------------------------------------------------------------------------------------

\subsection{Dati e Analisi}
In questa sezione si vuole inizialmente rappresentare le misure acquisite in laboratorio riportandole in un grafico
esplorativo di $V_{\text{out}}$ contro $V_{\text{in}}$, per cercare di estrarre informazioni di carattere generale sui
dati a disposizione: per ogni misura di un massimo di $V_{\text{in}}$ si associa il corrispettivo massimo di
$V_{\text{out}}$ (analogo per i minimi). Successivamente, si vuole invece caratterizzare la linearità dell'amplificatore
operazionale ed il guadagno del circuito in termini statistici, focalizzandosi sullo studio di interpolazioni lineari.
Rappresentando in un grafico $V_{\text{out}}$ contro $V_{\text{in}}$, infatti, è possibile ricavare il valore assoluto
delll'amplificazione come il coefficiente angolare di una retta che interpola i dati: dalla bontà del fit e
dall'andamento dei residui si riesce inoltre a studiare le proprietà di linearità del sistema in questione. 


%----------------------------------------------------------------------------------------
%	DATISET
%----------------------------------------------------------------------------------------

\subsubsection{Dataset}
Si riportano in  \autoref{i:opamp_eda} le misure acquisite con l'oscilloscopio alle quali si associa l'errore dato da

\begin{equation}\label{e:osc}
	\sigma_{V} = \sqrt{ (\sigma_{l}\times\text{V/div})^2 + (\sigma_{g}\times\text{measure})^2 }
\end{equation}

\noindent dove $\sigma_{l}=0.04$ e $\sigma_{g}=1.5\%$ rappresentano l'incertezza di lettura e di guadagno associati
all'oscilloscopio, V/div rappresenta la scala di acquisizione della misura, ovvero quanti Volt sono rappresentati in una
divisione dello schermo dell'oscilloscopio, mentre "measure" rappresenta la misura stessa.

%\begin{table}[H]
%	\centering
%	\footnotesize
%	\begin{tabular}{x{2cm} x{3cm} x{3.2cm} x{3cm} x{3.2cm}} 
%
%		\toprule[0.5px]
%		\toprule[0.1px]
%		
%		\multicolumn{5}{c}{Misure Acquisite con l'Oscilloscopio}\tn
%		\midrule[0.1px]
%
%		\multicolumn{5}{c}{Misure dei Massimi}\tn
%
%		\addlinespace
%		
%		$V\text{{pp}}_{\text{gen}}$ (V)& $V_{\text{in}}$ (V)& Scala $V_{\text{in}}$ (V/div)& $V_{\text{out}}$ (V)& Scala
%		$V_{\text{out}}$ (V/div)\tn
%		
%		\addlinespace
%		
%		$	0.20		$&$	0.106 \pm	0.003	$&$	0.050	$&$	1.00 \pm	0.02	$&$	0.324	$\tn
%		$	0.50		$&$	0.252 \pm	0.006	$&$	0.100	$&$	2.48 \pm	0.05	$&$	1.00	$\tn
%		$	0.80		$&$	0.400 \pm	0.010	$&$	0.200	$&$	4.00 \pm	0.10	$&$	2.00	$\tn
%		$	1.00		$&$	0.496 \pm	0.011	$&$	0.200	$&$	4.96 \pm	0.11	$&$	2.00	$\tn
%		$	1.50		$&$	0.744 \pm	0.014	$&$	0.200	$&$	7.44 \pm	0.14	$&$	2.00	$\tn
%		$	1.80		$&$	0.907 \pm	0.019	$&$	0.324	$&$	8.98 \pm	0.19	$&$	3.40	$\tn
%		$	2.00		$&$	1.01 \pm	0.02	$&$	0.324	$&$	9.9  \pm	0.2		$&$	3.40	$\tn
%		$	2.30		$&$	1.16 \pm	0.02	$&$	0.376	$&$	11.4 \pm	0.2		$&$	3.80	$\tn
%		$	2.60		$&$	1.29 \pm	0.03	$&$	0.436	$&$	13.0 \pm	0.3		$&$	4.52	$\tn
%		$	3.00		$&$	1.50 \pm	0.03	$&$	0.480	$&$	14.4 \pm	0.3		$&$	4.52	$\tn
%		$	3.20		$&$	1.61 \pm	0.03	$&$	0.630	$&$	14.7 \pm	0.3		$&$	5.60	$\tn	
%		$	3.50		$&$	1.77 \pm	0.04	$&$	0.660	$&$	14.7 \pm	0.3		$&$	5.60	$\tn
%		
%	
%		\addlinespace
%
%		\midrule[0.1px]
%		
%		\multicolumn{5}{c}{Misure dei Minimi}\tn
%
%		\addlinespace
%		
%		$V\text{{pp}}_{\text{gen}}$ (V)& $V_{\text{in}}$ (V)& Scala $V_{\text{in}}$ (V/div)& $V_{\text{out}}$ (V)& Scala
%		$V_{\text{out}}$ (V/div)\tn
%
%		\addlinespace
%
%		$	0.20		$&$	-0.102 \pm	0.003	$&$	0.050	$&$	-0.97 \pm	0.02	$&$	0.324	$\tn
%		$	0.50		$&$	-0.252 \pm	0.006	$&$	0.100	$&$	-2.48 \pm	0.05	$&$	1.00	$\tn
%		$	0.80		$&$	-0.400 \pm	0.010	$&$	0.200	$&$	-3.92 \pm	0.10	$&$	2.00	$\tn
%		$	1.00		$&$	-0.496 \pm	0.011	$&$	0.200	$&$	-4.96 \pm	0.11	$&$	2.00	$\tn
%		$	1.50		$&$	-0.736 \pm	0.014	$&$	0.200	$&$	-7.36 \pm	0.14	$&$	2.00	$\tn
%		$	1.80		$&$	-0.881 \pm	0.019	$&$	0.324	$&$	-8.98 \pm	0.19	$&$	3.40	$\tn
%		$	2.00		$&$	-0.98  \pm	0.02	$&$	0.324	$&$	-10.0 \pm	0.2		$&$	3.40	$\tn
%		$	2.30		$&$	-1.13  \pm	0.02	$&$	0.376	$&$	-11.5 \pm	0.2		$&$	3.80	$\tn
%		$	2.60		$&$	-1.29  \pm	0.03	$&$	0.436	$&$	-13.0 \pm	0.3		$&$	4.52	$\tn
%		$	3.00		$&$	-1.48  \pm	0.03	$&$	0.480	$&$	-14.1 \pm	0.3		$&$	4.52	$\tn
%		$	3.20		$&$	-1.59  \pm	0.03	$&$	0.630	$&$	-14.8 \pm	0.3		$&$	5.60	$\tn	
%		$	3.50		$&$	-1.72  \pm	0.04	$&$	0.660	$&$	-14.8 \pm	0.3		$&$	5.60	$\tn
%		
%		
%		\bottomrule[0.5px]
%		
%	\end{tabular}
%	\caption{Vengono rappresentate in tabella le misure sperimentali acquisite con i cursori dell'oscilloscopio 
%				con l'incertezza ad esse associata e la scala di acquisizione della misura.}
%	\label{t:osc_measures}
%\end{table}

\begin{figure}[H]
	\centering
	\includegraphics[width=15cm]{../Plots/Report_Plots/opamp_plot_alldata_eda.png}
	\caption{\small Grafico delle coppie ($V_{\text{in}}$, $V_{\text{out}}$) interpolate linearmente da una retta
	del tipo $y=a+bx$ con relativo grafico dei residui.}
	\label{i:opamp_eda}
\end{figure}

\noindent  Si vuole inizialmente far notare che i valori di $V_{\text{in}}$ sono conformi a quanto impostato sul
generatore: questo è sicuramente indice di una corretta acquisizione del segnale in ingresso e di una corretta
configurazione del generatore (\textit{modalità "50 Ohm"}) e dell'oscilloscopio (\textit{attenuazione sonda 10X}).
Osservando invece i valori di $V_{\text{out}}$ si nota un'amplificazione conforme alle aspettative (circa un fattore
10). Inoltre si osserva come le ultime misure, cioè quelle con tensione nominale $V\text{pp}_{\text{gen}}$ maggiore,
tendano a stabilizzarsi attorno a circa $V_{\text{sat}}=\pm15\,\si{\volt}$, ovvero la tensione massima che
l'amplificatore operazionale può fornire in output. Come da aspettative, riportate in \autoref{s:spice}, questo fenomeno
di stabilizzazione attorno a $V_{\text{sat}}$ inizia a manifestarsi attorno ad una tensione erogata dal generatore di
circa $V\text{pp}_{\text{gen}}=3\,\si{\volt}$. Dal grafico dei residui si può osservare lo stesso fenomeno: la zona
centrale risulta essere distribuita ragionevolmente attorno allo zero, mentre gli estremi tendono a distanziarsi anche
notevolmente. Da questo si deduce dunque che i tre punti finali di massimo e di minimo sono da considerarsi degli
outliers rispetto al trend lineare delle misure rimanenti: al fine di caratterizzare la linearità dell'amplificatore
operazionale e di calcolare l'amplificazione del circuito, dunque, gli outliers non verranno considerati.

%----------------------------------------------------------------------------------------
%	PRELIMIARY FIT
%----------------------------------------------------------------------------------------

\subsubsection{Interpolazioni Preliminari}\label{s:pre} Si procede ora considerando il campione di misure dei massimi ed
il campione di misure dei minimi separatamente, in quanto a priori non si ha la certezza che queste risentano della
stessa amplificazione e che non sia presente una sistematica di offset/shift verticale tra i due dataset. Si cercherà in
seguito di caratterizzare l'accordo tra i due dataset studiando la compatibilità tra i coefficienti angolari e tra le
intercette della retta interpolante. Osservando le misure in  \autoref{i:opamp_eda} si nota come le incertezze su
$V_{\text{in}}$ siano generalmente un ordine di grandezza inferiori rispetto a quelle su $V_{\text{out}}$: le prime non
sono quindi trascurabili rispetto alle seconde. Per tenere conto dell'incertezza su $V_{\text{in}}$, ci si propone
allora di effettuare un fit preliminare, nel quale si considerano unicamente gli errori su $V_{\text{out}}$, per stimare
un coefficiente angolare $m$. Questo viene poi utilizzato per proiettare gli errori di $V_{\text{in}}$ lungo l'asse
delle ordinate secondo 

\begin{equation}\label{e:proj}
	\sigma_{y} = \sqrt{	\sigma_{V_{\text{out}}}^2	+	m^2	\sigma_{V_{\text{in}}}^2	}
\end{equation}

\noindent I coefficienti angolari di interesse sono dunque riportati in  \autoref{t:pre_slopes}.

\begin{table}[H]
	\small
	\centering
	\begin{tabular}{x{4cm} x{4cm}} 

		\toprule[0.5px]
		\toprule[0.1px]
		
		\multicolumn{2}{c}{Coefficienti Angolari Preliminari}\tn
		\midrule[0.1px]

		Campione di Massimi & Campione di Minimi \tn

		\addlinespace
		
		$m=10.02\pm0.09$ & $m=10.16\pm0.09$ \tn
		
		\bottomrule[0.5px]
		
	\end{tabular}
	\caption{\small Valori dei coefficienti angolari restituiti dalle interpolazioni preliminari.}
	\label{t:pre_slopes}
\end{table}	

%----------------------------------------------------------------------------------------
%	LINEARITA E AMPLIFICAZIONE
%----------------------------------------------------------------------------------------


\subsubsection{Linearità e Amplificazione}
Alla luce di quanto trovato nella sezione precedente, si ripetono le interpolazioni lineari associando ai punti un
errore dato da  \autoref{e:proj} ed i parametri restituiti dai fit sono riportati in  \autoref{t:opamp_fitres_max_min}.

\begin{table}[H]
	\centering
	\small
	\begin{tabular}{x{3cm} x{3cm} x{3cm} x{3cm}} 

		\toprule[0.5px]
		\toprule[0.1px]
		
		\multicolumn{4}{c}{Fit Parameters}\tn
		\midrule[0.1px]

		\multicolumn{4}{c}{Campione di Massimi}\tn

		\addlinespace
		
		Offset (V) & Slope & $\chi^2$/ndf & $\sigma_{\text{posteriori}}$ (V)\tn

		\addlinespace

		$-0.06\pm0.04$ & $10.02\pm0.14$ & $0.98/7$ & $0.10$ \tn

		\midrule[0.1px]
		
		\multicolumn{4}{c}{Campione di Minimi}\tn

		\addlinespace
		
		Offset (V) & Slope & $\chi^2$/ndf & $\sigma_{\text{posteriori}}$ (V) \tn

		$0.07\pm0.04$ & $10.16\pm0.14$ & $0.67/7$ & $0.07$ \tn



		\bottomrule[0.5px]
		
	\end{tabular}
	\caption{\small Parametri della retta interpolante, il valore del $\chi^2$ associato al fit 
	e l'errore a posteriori relativo alla distribuzione dei dati.}
	\label{t:opamp_fitres_max_min}
\end{table}	

\noindent Dai parametri presentati in  \autoref{t:opamp_fitres_max_min} si riescono ad estrarre numerose informazioni
riguardo ai due campioni di dati. Inizialmente, si vuole far notare come i due coefficienti angolari siano in ottima
compatibilità tra loro: $\lambda=0.7$. Da questo si può assumere che i due dataset risentano della stessa amplificazione
$G$, come da aspettative. Successivamente, si può notare invece che le due intercette delle rette interpolanti sono in
leggera compatibilità con lo zero ($\lambda \approx 1.5$), mentre tra loro presentano una compatibilità $\lambda=2.4$,
che fa sorgere l'idea di una possibile sistematica di shift verticale tra i due dataset (computando la differenza tra le
due intercette si trova uno sfalsamento $d=0.13 \pm 0.05 \,\si{\volt}$). Osservando poi il valore del $\chi^2$, si
ritrova in per entrambi i campioni $\chi^2/\nu<1$ (con $\nu\equiv\text{ndf}$ il numero di gradi di libertà, che coincide
con il valore di aspettazione $E(\chi^2)$). Ricordando che le incertezze sul guadagno verticale dell'oscilloscopio sono
almeno parzialmente correlate, gli errori associati alle misure sono tra loro correlati: questo spiega i valori di
$\chi^2$ eccessivamente ridotti. Un'interpolazione di dati con incertezze correlate restituisce parametri con errori
sottostimati, in quanto il fit non tiene conto della correlazione tra incertezze delle misure. Si può dunque assumere
che i parametri \textit{slope} e \textit{offset} riportati in \autoref{t:opamp_fitres_max_min} presentino in realtà una
compatibilità maggiore, proprio a causa di una possibile sottostima dell'errore sui parametri. Si vuole allora assumere
che i due campioni risentano della stessa amplificazione e che non siano tra loro sfalsati verticalmente in modo
significativo, segue un tentativo di "unificazione" del campione di dati ed un'interpolazione lineare unica che tenga
conto sia dei massimi che dei minimi.
Il grafico rappresentante i due dataset unificati con relativa interpolazione lineare è mostrato in
\autoref{i:opamp_all_proj}. 

\begin{figure}[H]
	\centering
	\includegraphics[width=15cm]{../Plots/Report_Plots/opamp_plot_all_projected.png}
	\caption{\small A sinistra: grafico rappresentante il dataset dei massimi ed il dataset dei minimi uniti assieme, 
	con relativa retta interpolante e parametri del fit. A destra: grafico dei residui $V_{\text{out}}-\text{fit}$.}
	\label{i:opamp_all_proj}
\end{figure}

\noindent Si osserva inizialmente che l'intercetta della retta interpolante è ora ben compatibile con zero, mentre il
coefficiente angolare presenta un errore relativo $\sigma_{b}/b=0.7\%$, che si può continuare ad assumere sottostimato:
la correlazione tra gli errori di scala, infatti, si può notare chiaramente dall'andamento "a farfalla" delle barre
d'errore nel grafico dei residui. Il valore del $\chi^2$ migliora leggermente rispetto alle interpolazioni dei dataset
separati: la compatibilità con il valore di aspettazione risulta essere $Z=1.4$. L'errore a posteriori, inoltre, si
trova in una zona intermedia rispetto alla gamma di errori associati alle misure: non potendo eliminare la correlazione
tra le incertezze si può affermare dunque che l'errore è in media ben stimato e l'oscilloscopio lavora entro le
specifiche. I residui, infatti, si posizionano tutti entro il loro errore, alcuni anche abbondantemente. Focalizzandosi
ora sulla stima del coefficiente angolare si nota che questo, $m=9.93\pm 0.07$, pur essendo ben compatibile con i
risultati esposti in  \autoref{t:opamp_fitres_max_min} relativi ai fit dei due campioni di misure considerati
separatamente, si trova essere sensibilmente minore di entrambi: ci si sarebbe aspettato, invece, di trovare un valore
intermedio unificando i due campioni di misure. Osservando poi il grafico dei residui, si può notare un andamento
leggermente anomalo, quasi parabolico, avente concavità rivolta verso il basso. Si ipotizza dunque che l'assunzione
fatta in precedenza riguardo la presenza di una sistematica di offset/shift verticale tra i due dataset trascurabile
necessiti di essere rivisitata. Per approfondire maggiormenta la questione, si decide di computare le grandezze "picco
picco" delle tensioni in ingresso $V_{\text{in}}$ e in uscita $V_{\text{out}}$ secondo
$V_{\text{pp}}=V^{\text{max}}-V^{\text{min}}$. Per quanto riguarda l'errore da associare alle grandezze picco picco, si
ricorda che l'oscilloscopio misura la differenza $\Delta$ tra i due cursori con una precisione ancora maggiore rispetto
alla singola misura. Si decide dunque di non aggiungere il fattore moltiplicativo $\sqrt{2}$ alla propagazione
presentata in  \autoref{e:osc}, al fine di evitare di sovrastimare eccessivamente l'errore. Si procede ora esattamente
come mostrato in  \autoref{s:pre}, effettuato inizialmente un fit lineare preliminare considerando solo gli errori su
$Vpp_{\text{out}}$ e, utilizzando il coefficiente angolare restituito da tale interpolazione, si prosegue proiettando
gli errori secondo  \autoref{e:proj}. Si ripete quindi il fit, che viene rappresentato in \autoref{i:opamp_pp}.

\begin{figure}[H]
	\centering
	\includegraphics[width=15cm]{../Plots/Report_Plots/opamp_plot_pp_projected.png}
	\caption{A sinistra: grafico rappresentante il dataset delle grandezze picco picco, 
	con relativa retta interpolante e parametri del fit. A destra: grafico dei residui $V\text{pp}_{\text{out}}-\text{fit}$.}
	\label{i:opamp_pp}
\end{figure}

\noindent Si nota immediatamente, osservando il grafico dei residui, come ora l'andamento anomalo è del tutto assente ed
i punti si distribuiscono in modo ottimale attorno allo zero. Rimane, chiaramente, il tipico andamento crescente delle
barre d'errore, indice che le incertezze continuano a risentire della correlazione tra esse. Il valore del $\chi^2$ è
decisamente basso rispetto al numero di gradi di libertà, come suggerito dal grafico dei residui in cui si nota
chiaramente come la distanza punto-retta sia ampiamente compresa entro la barra d'errore del dato. L'errore a posteriori
è appena maggiore dell'incertezza associata al primo punto, mentre diventa notevolmente inferiore rispetto ai punti
finali. Il valore dell'intercetta, scarsamente compatibile con lo zero, suggerisce una conferma all'ipotesi un una
sistematica di offset/shift verticale tra i due dataset non trascurabile. Il coefficiente angolare, invece, è
perfettamente in linea con i parametri ottenuti considerando i due dataset separatamente: calcolando la media pesata dei
due, infatti, si trova $\langle m\rangle_{\text{max, min}}=10.09 \pm 0.10$ e risulta avere una compatibilità
estremamente elevata con il coefficiente angolare riguardante il dataset delle grandezze picco picco ($\lambda = 0.01$).
Si assume dunque che questi due valori (media pesata dei coefficienti angolari $\langle m\rangle_{\text{max, min}}$ e
coefficiente angolare del campione di grandezze picco picco $m_{\text{pp}}$) rappresentino una soddisfacente stima
dell'amplificazione \textit{G} del circuito. Per quanto riguarda la linearità dell'amplificatore operazionale, invece, i
valori estremamente ridotti del $\chi^2$ non permettono nè di confermare l'ipotesi di linearità nè di poterla rigettare.
Si ripone allora maggior attenzione alla distribuzione delle misure attorno alla retta (o meglio alla distribuzione dei
residui attorno allo zero) che si ritiene invece, in questa occasione, determinante: il campione di misure picco picco
( \autoref{i:opamp_pp}) suggerisce una soddisfacente distribuzione lineare dei dati. 


%----------------------------------------------------------------------------------------
%	CONFRONTO STIME DI G
%----------------------------------------------------------------------------------------

\subsubsection{Confronto tra Stime di G}
Si vuole ora esporre e confrontare le stime dell'amplificazione del circuito, rappresentando i valori del guadagno
\textit{G} in \autoref{i:opamp_comp}.

\begin{wrapfigure}{L}{0.5\textwidth}
	\centering
	\includegraphics[width=0.5\textwidth]{../Plots/Report_Plots/opamp_comp_BIG.png}
	\caption{\footnotesize Stime di G. Da sinistra: 1) partendo dalle misure dirette delle resistenze; 2) come
	coefficiente angolare del dataset di massimi; 3) come coefficiente angolare del dataset di minimi; 4) come
	coefficiente angolare del dataset unificato; 5) come media pesata di 2 e 3; 6) come coefficiente angolare del
	dataset delle grandezze picco picco.}
	\label{i:opamp_comp}
\end{wrapfigure}

\noindent Partendo dal primo punto a sinistra, cioè la stima di \textit{G} tramite le misure dirette delle resistenze
$R_{\text{f}}$ e $R_{1}$ (riportate in \autoref{t:direct_measures}), si nota come questo presenti un errore nettamente
inferiore a confronto con le rimanenti stime. Quest'ultime risultano essere quantità compatibili con 1)
$G=R_{\text{f}}/R_{1}$, ad eccezione di 4) quella ottenuta considerando assieme sia i massimi sia i minimi ($\lambda =
3.7$). In particolare, si può osservare come la media pesata 5) tra le stime dell'amplificazione ottenute considerando i
campioni separati e la stima ottenuta con le grandezze picco picco 6) si trovino in eccellente accordo: si può concludere
dunque che, eliminando la sistematica di offset/shift verticale tra i due dataset (sia attraverso grandezze picco picco,
sia considerando la media pesata dei risultati ottenuti dai campioni separati), la stima dell'amplificazione del
circuito risulta essere compatibile con le aspettative preliminari. Si assume in ogni caso che l'errore su \textit{G}
sia sottostimato a causa della correlazione tra errori di scala dell'oscilloscopio: si preferisce dunque la stima
ritrovata considerando le tensioni picco picco 6), in quanto presenta un errore relativo leggermente maggiore.


%----------------------------------------------------------------------------------------
%	CIRCUITO DERIVATORE
%----------------------------------------------------------------------------------------

\section{Circuito Derivatore}

\cleardoublepage
%----------------------------------------------------------------------------------------
%	ARDUINO
%----------------------------------------------------------------------------------------
\cleardoublepage
\section{Arduino}
In questa sezione si vuole effettuare una calibrazione della scheda Arduino Due. In particolare, si vuole quantificare
il sampling rate dell'ADC della scheda e determinare la funzione di calibrazione in tensione, ovvero $V = a
+ b \cdot \text{ADC}$ dove $V$ è il valore in Volt del segnale, $\text{ADC}$ è la tensione in ADC counts acquisita da
Arduino, mentre $a$ e $b$ sono i parametri di calibrazione. 

%----------------------------------------------------------------------------------------
%	SAMPLING RATE 
%----------------------------------------------------------------------------------------

\subsection{Sampling Rate}
Si comincia configurando il segnale di trigger. Si imposta quindi nel canale CH2 del generatore un impulso quadrato di
durata $10\,\si{\us}$, frequenza $1\,\si{k\hertz}$ e altezza $2\,\si{\volt}$ a partire dallo zero. Sul canale CH1 del
generatore, invece, si imposta un'onda quadra di ampiezza $1\,\si{\volt}$ partendo da zero con frequenza
$5\,\si{k\hertz}$. Conoscendo il periodo dell'onda quadra in ingresso ($T=1/f$), il sampling rate viene computato come
$S = N / T = N \, f$ con $N$ il numero di misure acquisite in un periodo. Per calcolare $N$ viene computata la derivata
numerica della forma d'onda: questa presenterà dei picchi positivi quando la funzione passa da zero a $1\,\si{\volt}$ e
picchi negativi quando scende da $1\,\si{\volt}$ a zero. Il numero di acquisizioni in un periodo sarà allora il numeri
di punti compresi tra due picchi positivi della funzione derivata. Si trova allora un sampling rate $S = 955000 s^{-1}$,
ovvero 955000 acquisizioni al secondo.


%----------------------------------------------------------------------------------------
%	CALIBRAZIONE IN TENSIONE
%----------------------------------------------------------------------------------------

\subsection{Calibrazione in Tensione}
Si vuole ora verificare la linearità dell'ADC interno alla scheda e stimare i parametri $a$, $b$ della funzione di
calibrazione, in quanto si è interessati a convertire il segnale acquisito da ADC counts in Volt. Si acquisiscono allora
diverse forme d'onda facendo variare la tensione del generatore, avendo cura di misurare il segnale erogato con i
cursori dell'oscilloscopio, in quanto può non essere esattamente uguale a quello nominale indicato dal generatore. Si
rappresentano in grafico i valori di tensione $V$ misurati sperimentalmente contro la media dei punti appartenenti ai
picchi della relativa forma d'onda (si decide di non considerare unicamente il massimo della forma in quanto è possibile
si tratti di una fluttuazione). Effettuando poi un'interpolazione lineare si ricavano l'offset (cioè
quanti Volt corrispondono allo zero dell'ADC) ed il coefficiente angolare (cioè come scalano i Volt rispetto all'ADC).

\begin{figure}[H]
	\centering
	\includegraphics[width=15cm]{../Arduino/Plots/calib_function.png}
	\caption{Grafico di calibrazione in tensione e relativo grafico dei residui.}
	\label{i:ar_calib}
\end{figure}

%\noindent Osservando il grafico dei residui, si nota immediatamente come sia presente un andamento anomalo dei punti a
%tensioni maggiori: la scheda risponde cioè in modo leggermente diverso a seconda della tensione in ingresso ed il
%fenomeno tende ad accentuarsi per segnali oltre circa $2\,\si{\volt}$. Questo è molto probabilmente dovuto al circuito
%di protezione dei pin di ingresso (limitatore di tensione a diodi, utile per evitare di bruciare la scheda) che ne
%altera la risposta a tensioni oltre i $2\,\si{\volt}$. Si prova allora ad effettuare nuovamente l'interpolazione
%rimuovendo i punti relativi a tensioni in ingresso maggiori di $2\,\si{\volt}$: nonostante l'andamento dei residui
%migliori, anche il primo punto (tensione in ingresso pari a $200\,\si{\mV}$) si trova essere fuori trend. Si ottiene
%invece un'ottima linearità della scheda se utilizzata con tensioni in ingresso tra $500\,\si{\mV}$ e $1.8\,\si{\volt}$,
%oppure se utilizzata con tensione in ingresso tra $1.8\,\si{\volt}$ e $2.5\,\si{\volt}$
%Si consiglia dunque di utilizzare la scheda Arduino Due in tale range di tensioni, per il quale si ottiene una funzione
%di calibrazione $V = a + b\,\text{ADC}$ con $a = -0.59 \pm 0.02 \, \si{\V}$ e $b = 0.776 \pm 0.013
%\,\si{\mV}/\text{a.u.}$.

\noindent Osservando il grafico dei residui, si nota un marcato andamento anomalo dei punti a tensioni maggiori, oltre i
$2\,\si{\volt}$: la scheda, cioè, risponde in modo leggermente diverso a seconda della tensione in ingresso. Questo è,
molto probabilmente, dovuto al circuito di protezione dei pin di ingresso (limitatore di tensione a diodi, utile per
evitare di bruciare la scheda) che ne altera la risposta avvicinandosi a tensioni pericolose. Si prova allora ad
effettuare nuovamente l'interpolazione rimuovendo i punti relativi a tensioni in ingresso maggiori di $2\,\si{\volt}$:
nonostante l'andamento dei residui migliori, anche il primo punto (tensione in ingresso pari a $200\,\si{\mV}$) si trova
essere fuori trend. Si ottengono quindi due zone in cui la linearità dell'ADC risulta essere ottimale: la prima tra
$500\,\si{\mV}$ e $1.8\,\si{\volt}$ (parametri di calibrazione: $a = -0.59 \pm 0.02 \, \si{\V}$ e $b = 0.776 \pm 0.013
\,\si{\mV}/\text{a.u.}$) mentre la seconda tra $1.8\,\si{\volt}$ e $2.5\,\si{\volt}$ (parametri di calibrazione: $a =
-0.44 \pm 0.15 \, \si{\V}$ e $b = 0.73 \pm 0.04 \,\si{\mV}/\text{a.u.}$).



%----------------------------------------------------------------------------------------
%	CONCLUSIONI
%----------------------------------------------------------------------------------------

\section{Conclusioni}

\end{document}