\documentclass[a4paper,11pt]{article}
\usepackage[T1]{fontenc}
\usepackage[utf8]{inputenc}
\usepackage[italian]{babel}

%packages
\usepackage[table]{xcolor}
\usepackage{lipsum}	
\usepackage{gensymb}
\usepackage{amsmath}
\usepackage{amssymb}
\usepackage{bm}
\usepackage{tikz}
\usepackage{hhline}
\usepackage{listings}
\usepackage{xcolor}
\lstset {language=C++, backgroundcolor=\color{white!5}, % set backgroundcolor
	basicstyle=\footnotesize,% basic font setting
}

\newcommand{\Mypm}{\mathbin{\tikz [x=1.4ex,y=1.4ex,line width=.1ex] \draw (0.0,0) -- (1.0,0) (0.5,0.08) -- (0.5,0.92) (0.0,0.5) -- (1.0,0.5);}}

%margini
\usepackage[margin=2cm,includefoot]{geometry}
\usepackage{fancyhdr}
\usepackage{rotating}
%link dell'indice ecc
\usepackage[hidelinks]{hyperref}

\usepackage[none]{hyphenat}%non spezza le parole nelle tabelle
\usepackage{array}
%\newcolumntype{£}{>{\global\let\currentrowstyle\relax}} \newcolumntype{§}{>{\currentrowstyle}} \newcolumntype{\rowstyle}[1]{\gdef\currentrowstyle{#1}
%#1 \ignorespaces}

%graphics
\usepackage{graphicx}
\usepackage{float}
\usepackage{wrapfig}
\usepackage{caption}
\usepackage{subcaption}


%pagestyle
\pagestyle{fancy}
\fancyhead{}
\fancyfoot{}
\fancyfoot[R]{\thepage}
\renewcommand{\headrulewidth}{0pt}

%math
\usepackage{xfrac}
\usepackage{amssymb}
	\newcommand{\restr}[2]{{% we make the whole thing an ordinary symbol
		\left.\kern-\nulldelimiterspace % automatically resize the bar with \right
		#1 % the function
		
		\right|_{#2} % this is the delimiter
}}
\usepackage{multicol}
\newcommand{\tnhl}{\tabularnewline\hline}
\newcommand{\tn}{\tabularnewline}
\newcolumntype{x}[1]{%
	>{\centering\hspace{0pt}}p{#1}}%

\usepackage[toc,page]{appendix}
\usepackage{booktabs}
\usepackage{siunitx}
\usepackage{multirow}


\begin{document}
	\def\subsectionautorefname{Sezione} \def\subsubsectionautorefname{Sezione}




	%----------------------------------------------pagina del titolo
	\begin{titlepage}
		\begin{center}
			\Huge{\bfseries Laboratorio di Fisica}\\
				
			\LARGE Docenti: Prof. A. Garfagnini - Prof. M. Lunardon \\
			\Large Corso di Laurea in Fisica\\
			\Large Canale 1 A-L\\
			\Large Anno Accademico 2020/2021\\
			[1cm] \line(1,0){400}\\
			[2cm]
				
			\textsc{\huge{\bfseries  Esperienza di Laboratorio $\text N^o$ 1}}\\
			\huge{Amplificatori Operazionali}\\
			[2mm] \line(1,0){300}\\
			[10.5cm]
					
		\end{center}
		
			
			\textsc{\Large Turno T2}\\
			[0.5cm] \textsc{\large {\bfseries Lai Nicolò}} \\ 
			\indent\large 1193976 \\ 
			\indent\large nicolo.lai@studenti.unipd.it\\
			
				
				
		\begin{flushright}
				\textsc{\Large Data di consegna:}\\
				\textsc{\large ?/11/2020}					
		\end{flushright}
				
	\end{titlepage}
\cleardoublepage
	%-----------------------------------fine pagina titolo

\section{Obiettivo}
Misura dell'amplificazione \textit{A} di un circuito con amplificatore operazionale. Misura della frequenza di taglio di un circuito derivatore con
amplificatore operazionale.

\section{Apparato Sperimentale}
\begin{itemize}
	\item \textbf{Oscilloscopio} (Tektronix TBS1102B): Lo strumento presenta un'accuratezza sul guadagno verticale pari al 3\% del valore letto
	(errore massimo) ed è \textit{generalmente} il contributo più significativo. L'incertezza di guadagno sui tempi si assume trascurabile.
	L'accuratezza che tiene conto degli effetti di risoluzione e imprecisione della traccia è di 1/10 di divisione su tutta la scala di lettura
	(errore massimo), uguale sia per le tensioni sia per i tempi.
	\item \textbf{Generatore di funzioni} (Tektronix AFG1022)
	\item \textbf{Multimetro digitale} (Metrix MTX3292): Si riporta l'accuratezza dello strumento, per misure di resistenza e di capacità, relativa
	unicamente ai fondoscala utilizzati nell'esperienza.

	\begin{table}[H]
		\centering
		\begin{tabular}{x{2cm} x{3cm} x{3cm} }
			\toprule[0.5px]\toprule[0.1px]
			
			\multicolumn{3}{c}{Accuratezza Metrix MTX3292}\tn
			\midrule[0.1px]
			
			F.S. & Precisione & Risoluzione \tn
			
			\addlinespace
			
			1   \si{k\ohm} & 0.10\% + 8  & 0.01 \si{\ohm}  \tn
			10  \si{k\ohm} & 0.07\% + 8  & 0.1  \si{\ohm}  \tn
			100 \si{k\ohm} & 0.07\% + 8  & 1    \si{\ohm}  \tn

			
			\addlinespace

			1000 \si{p\farad}         & 2.5\% + 15  & 1 \si{p\farad}   \tn
			
			\bottomrule[0.5px]
			
			
		\end{tabular}
		\caption{Per i fondoscala indicati (prima colonna) viene mostrata la precisione (contributo di scala in percentuale e contributo di lettura
		sul digit meno significativo) e la risoluzione dello strumento (seconda e terza colonna)}
		\label{t:metrix}
	\end{table}	 

	\item \textbf{Alimentatore di tensione continua}
	\item \textbf{Circuito integrato TL082C}
	\item \textbf{Scheda Arduino Due}
\end{itemize}

	

 






















\end{document}