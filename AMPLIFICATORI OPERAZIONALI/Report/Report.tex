\documentclass[a4paper,10pt]{article}
\usepackage[T1]{fontenc}
\usepackage[utf8]{inputenc}
\usepackage[italian]{babel}

%packages
\usepackage[table]{xcolor}
\usepackage{lipsum}	
\usepackage{gensymb}
\usepackage{amsmath}
\usepackage{amssymb}
\usepackage{bm}
\usepackage{tikz}
\usepackage{hhline}
\usepackage{listings}
\usepackage{xcolor}
\lstset { 
	language=C++,
	backgroundcolor=\color{white!5}, % set backgroundcolor
	basicstyle=\footnotesize,% basic font setting
}

\newcommand{\Mypm}{\mathbin{\tikz [x=1.4ex,y=1.4ex,line width=.1ex] \draw (0.0,0) -- (1.0,0) (0.5,0.08) -- (0.5,0.92) (0.0,0.5) -- (1.0,0.5);}}

%margini
\usepackage[margin=0.8in,includefoot]{geometry}
\usepackage{fancyhdr}
\usepackage{rotating}
%link dell'indice ecc
\usepackage[hidelinks]{hyperref}

\usepackage[none]{hyphenat}%non spezza le parole nelle tabelle
\usepackage{array}
%\newcolumntype{£}{>{\global\let\currentrowstyle\relax}}
%\newcolumntype{§}{>{\currentrowstyle}}
%\newcolumntype{\rowstyle}[1]{\gdef\currentrowstyle{#1} #1 \ignorespaces}

%graphics
\usepackage{graphicx}
\usepackage{float}
\usepackage{wrapfig}
\usepackage{caption}
\usepackage{subcaption}


%pagestyle
\pagestyle {fancy}
\fancyhead{}
\fancyfoot{}
\fancyfoot[R]{\thepage}
\renewcommand{\headrulewidth}{0pt}

%math
\usepackage{xfrac}
\usepackage{amssymb}
	\newcommand{\restr}[2]{{% we make the whole thing an ordinary symbol
		\left.\kern-\nulldelimiterspace % automatically resize the bar with \right
		#1 % the function
		
		\right|_{#2} % this is the delimiter
}}
\usepackage{multicol}
\newcommand{\tnhl}{\tabularnewline\hline}
\newcommand{\tn}{\tabularnewline}
\newcolumntype{x}[1]{%
	>{\centering\hspace{0pt}}p{#1}}%

\usepackage[toc,page]{appendix}
\usepackage{booktabs}
\usepackage{siunitx}
\usepackage{multirow}


\begin{document}
	\def\subsectionautorefname{Sezione}	
	\def\subsubsectionautorefname{Sezione}
	%pagina del titolo
	\begin{titlepage}
		\begin{center}
			\Huge{\bfseries Laboratorio di Fisica}\\
				
			\LARGE Docenti: Prof. A. Garfagnini - Prof. M. Lunardon \\
			\Large Corso di Laurea in Fisica\\
			\Large Canale 1 A-L\\
			\Large Anno Accademico 2020/2021\\
			[1cm]
			\line(1,0){400}\\
			[2cm]
				
			\textsc{\huge{\bfseries  Esperienza di Laboratorio $\text N^o$ 1}}\\
			\huge{Amplificatori Operazionali}\\
			[2mm]
			\line(1,0){300}\\
			[12cm]
					
		\end{center}
		
			
			\textsc{\Large Turno T2}\\
			[0.5cm]
			\textsc{\large {\bfseries Lai Nicolò}} \\ 
			\indent\large 1193976 \\ 
			\indent\large nicolo.lai@studenti.unipd.it\\
			
				
				
		\begin{flushright}
				\textsc{\Large Data di consegna:}\\
				\textsc{\large ?/11/2020}					
		\end{flushright}
				
	\end{titlepage}
\cleardoublepage


	
\end{document}