%%% use twocolumn and 10pt options with the asme2ej format
\documentclass[twocolumn,10pt]{asme2ej}

\usepackage{epsfig} %% for loading postscript figures
\usepackage{lipsum}
\usepackage{titling}

\pretitle{\begin{center}\linespread{1.2}\huge}
\posttitle{\par\end{center}\vspace{0.5em}}

%% The class has several options
%  onecolumn/twocolumn - format for one or two columns per page
%  10pt/11pt/12pt - use 10, 11, or 12 point font
%  oneside/twoside - format for oneside/twosided printing
%  final/draft - format for final/draft copy
%  cleanfoot - take out copyright info in footer leave page number
%  cleanhead - take out the conference banner on the title page
%  titlepage/notitlepage - put in titlepage or leave out titlepage
%  
%% The default is oneside, onecolumn, 10pt, final

\date{}
\title{{\huge\bfseries Laboratorio di Fisica} - {\LARGE A.A. 2020/2021} \\ 
    {\LARGE Docenti: A. Garfagnini - M. Lunardon} \\ {\Huge\bfseries Effetto Zeeman}}


%%% first author
\author{Cerrone Vanessa
    \affiliation{
    1200361\\
    vanessa.cerrone@studenti.unipd.it
    }	
}

%%% second author
%%% remove the following entry for single author papers
%%% add more entries for additional authors
\author{Cigagna Simone
    \affiliation{
	1193992\\
    simone.cigagna@studenti.unipd.it
    }	
}

%%% third author
%%% remove the following entry for single author papers
%%% add more entries for additional authors
\author{Lai Nicolò
    \affiliation{
	1193976\\
    nicolo.lai@studenti.unipd.it
    }	
}


\begin{document}

\maketitle    


% %%%%%%%%%%%%%%%%%%%%%%%%%%%%%%%%%%%%%%%%%%%%%%%%%%%%%%%%%%%%%%%%%%%%%%
\section{Introduzione}

\begin{itemize}
    \item Cos'è l'effetto Zeeman
    \item Lampada transizioni righe cose di fisica
    \item Come facciamo e cosa facciamo (riga proiezioni grafici background fit cose) 
    \item Cosa vogliamo trovare/stimare (R, Landè e cose polarizzazione)(specificare con che B)
\end{itemize}


% %%%%%%%%%%%%%%%%%%%%%%%%%%%%%%%%%%%%%%%%%%%%%%%%%%%%%%%%%%%%%%%%%%%%%%
\section{Spettro di emissione del Neon}

\begin{itemize}
    \item Calibrazione
    \item Plot dello spettro con Boff
    \item Dire che non cambia niente se Bon (solo intensità o rumore)
    \item Identificare la riga che poi analizziamo
\end{itemize}


% %%%%%%%%%%%%%%%%%%%%%%%%%%%%%%%%%%%%%%%%%%%%%%%%%%%%%%%%%%%%%%%%%%%%%%
\section{Potere risolvente dell'apparato}

\begin{itemize}
    \item Lo calcoliamo con Boff
    \item Formule (range utile ecc ecc) (approx luce radente) 
    \item Procedura di analisi (indipendenza statistica, picchi a triplette ecc ecc)
    \item Plot con tutti i picchi fittati + finestrella con lo zoom su una tripletta
    \item Stima di R (come quando perchè)
    \item Plot del trend per aberrazione 
\end{itemize}



% %%%%%%%%%%%%%%%%%%%%%%%%%%%%%%%%%%%%%%%%%%%%%%%%%%%%%%%%%%%%%%%%%%%%%%
\section{Fattore di Landè}

\begin{itemize}
    \item Lo calcoliamo con Bon lungo la direzione della radiazione (parlare dello splitting di Zeeman)
    \item Formule (range utile + conversione energia + fattore di Landè)
    \item Procedura di analisi (niente fit perchè non vengono)
    \item Plot con tutti i picchi + finestrella con lo zoom su una tripletta 
    \item Controllare che lo splitting Zeeman non subisca aberrazione pesante
\end{itemize}


% %%%%%%%%%%%%%%%%%%%%%%%%%%%%%%%%%%%%%%%%%%%%%%%%%%%%%%%%%%%%%%%%%%%%%%
\section{Campo magnetico ortogonale alla radiazione}

\begin{itemize}
    \item Un po' di fisica più in dettaglio rispetto all'introduzione (polarizzazione ecc) + aspettative "teoriche"
    (intensità picchi ecc)
    \item Plot istogrammi sovrapposti 
    \item Deduzione configurazioni polarimetro dal plot 
\end{itemize}



\end{document}
