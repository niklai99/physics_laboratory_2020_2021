%%% use twocolumn and 10pt options with the asme2ej format
\documentclass[twocolumn,10pt]{asme2ej}

\usepackage{epsfig} %% for loading postscript figures
\usepackage{subfigure}
\usepackage{titling}
\usepackage{siunitx}
\usepackage[hidelinks]{hyperref}
\usepackage{fancyhdr}
\usepackage{booktabs}
\pagestyle{fancy}
\fancyhead{}
\fancyfoot{}
\fancyfoot[R]{\thepage}

\def\figureautorefname{Fig.} 
\def\equationautorefname{Eq.} 
\def\sectionautorefname{Sez.} 
\def\subsectionautorefname{Sez.} 
\def\subsubsectionautorefname{Sez.}
\def\tableautorefname{Tab.}

\pretitle{\begin{center}\linespread{1.2}\huge}
\posttitle{\par\end{center}\vspace{0.5em}}

\abovedisplayshortskip=0pt
\belowdisplayshortskip=0pt
\abovedisplayskip=-5pt
\belowdisplayskip=5pt


%% The class has several options
%  onecolumn/twocolumn - format for one or two columns per page
%  10pt/11pt/12pt - use 10, 11, or 12 point font
%  oneside/twoside - format for oneside/twosided printing
%  final/draft - format for final/draft copy
%  cleanfoot - take out copyright info in footer leave page number
%  cleanhead - take out the conference banner on the title page
%  titlepage/notitlepage - put in titlepage or leave out titlepage
%  
%% The default is oneside, onecolumn, 10pt, final

\date{}
\title{{\huge\bfseries Laboratorio di Fisica} - {\LARGE A.A. 2020/2021} \\ 
    {\LARGE Docenti: A. Garfagnini - M. Lunardon} \\ {\Huge\bfseries Fotodiodo}}


%%% first author
\author{Cerrone Vanessa
    \affiliation{
    1200361\\
    vanessa.cerrone@studenti.unipd.it
    }	
}

%%% second author
\author{Cigagna Simone
    \affiliation{
	1193992\\
    simone.cigagna@studenti.unipd.it
    }	
}

%%% third author
\author{Lai Nicolò
    \affiliation{
	1193976\\
    nicolo.lai@studenti.unipd.it
    }	
}


\begin{document}


\maketitle    


% %%%%%%%%%%%%%%%%%%%%%%%%%%%%%%%%%%%%%%%%%%%%%%%%%%%%%%%%%%%%%%%%%%%%%%
\section{Introduzione}\label{s:introduzione}
Si vuole analizzare lo spettro dei fotoni emessi dall'Americio-241 con un rivelatore al Silicio tipo PIN,
dotato di preamplificatore di carica. L’hardware, cioè i moduli di elettronica, sono stati pre-impostati
in condizioni standard, con shaping time pari a $3\mu\si{\second}$, in modo da ottimizzare il rapporto
segnale rumore. Preliminarmente, tramite il software di acquisizione, si registra uno spettro per 
identificare i picchi principali, a 60keV e 14-18keV. \\
Nella \autoref{s:attenuazione} si analizzerà il picco a 59.5keV in presenza di materiali di diverso spessore,
al fine di calcolare i relativi coefficienti di assorbimento. Nella  \autoref{s:distanza}, si effettueranno 
misure al variare della distanza della sorgente, per verificare che i dati seguano l'andamento atteso. 
Un'analisi dettagliata dello spettro verrà presentata nella sezione \autoref{s:multipicco}. 

% scriviamo 60 59 59.5 keV ???



% %%%%%%%%%%%%%%%%%%%%%%%%%%%%%%%%%%%%%%%%%%%%%%%%%%%%%%%%%%%%%%%%%%%%%%
\section{Calibrazione e risoluzione energetica}\label{s:calibrazione}
% Scrivere della misura preliminare per verificare la precisione del picco a 60 keV??
% Calibrazione e calcolo della risoluazione ? 


% %%%%%%%%%%%%%%%%%%%%%%%%%%%%%%%%%%%%%%%%%%%%%%%%%%%%%%%%%%%%%%%%%%%%%%
\section{Coefficiente di assorbimento}\label{s:attenuazione}
Ci si propone di effettuare delle misure in presenza di materiali di diverso spessore, nello specifico
rame e argento, con lo scopo di calcolarne il coefficiente di assorbimento $\mu$, che si ricava dalla relazione:

\vspace{-15pt}
\begin{equation}
   \text{I}(x) = \text{I}_0 \text{e}^{-\mu x}
    \vspace{-5pt}
    \label{e:mu}
\end{equation}

dove I è l'intensità della radiazione incidente e x lo spessore attraversato. 
\\ Si inseriscono gli assorbitori di spessore variabile e si acquisiscono gli spettri per
un intervallo di tempo sufficiente a garantire una precisione migliore del 3\% sul picco a 59.5keV.
La precisione in percentuale si ottiene ricavando il numero di eventi N, cioè l'area, al di sotto del
picco di interesse...
Si calcola il rate degli eventi nel picco a 60 keV per tutte le misure effettuate come rapporto tra numero di 
eventi rilevati e tempo di acquisizione, che come prima è stato adattato in modo da avere precisioni di almeno il 3\%
Considerando la relazione \autoref{e:mu} si effettua un fit esponenziale del rate in funzione dello spessore del materiale, 
separatamente per rame e argento. Si sottolinea che il rapporto N/t rappresenta l'intensità della radiazione incidente
per unità di superficie: il rivelatore a disposizione ha un'area di 1 $\si{\centi\metre}^2$, dunque ok (???).

\begin{table}[t]
%%\renewcommand{\arraystretch}{1.3}
    \centering
    \resizebox{\linewidth}{!}{%
    \begin{tabular}{@{}cc|cc@{}}
    \toprule
    \multicolumn{2}{c}{\textbf{Ag}}                          & \multicolumn{2}{c}{\textbf{Cu}}                        \\ \midrule
    \multicolumn{1}{c}{Spessore [$\si{\micro\metre}$]} & \multicolumn{1}{c|}{Rate [Hz]} & \multicolumn{1}{c}{Spessore [$\si{\micro\metre}$]} & \multicolumn{1}{c}{Rate [Hz]} \\
    60                           & $5.4 \pm 0.1 $                    & 92                           & $6.97 \pm 0.13$                    \\
    120                          & $3.80 \pm 0.08 $                     & 184                          & $6.1  \pm 0.1$                    \\
    180                          & $2.42  \pm  0.07$                    & 276                          & $5.5 \pm 0.1$                      \\
    240                          & $1.93  \pm  0.06$                    & 368                          & $4.7 \pm 0.1$                      \\ \bottomrule
    \end{tabular}%
    }
    \caption{Dati fit esponenziale per il calcolo del coefficiente di assorbimento}
    \label{t:assorbimento}
    \end{table}

\clearpage
% %%%%%%%%%%%%%%%%%%%%%%%%%%%%%%%%%%%%%%%%%%%%%%%%%%%%%%%%%%%%%%%%%%%%%%
\section{Misure in funzione della distanza}\label{s:distanza}


\begin{table}[h ]
    \centering
    \resizebox{\linewidth}{!}{%
    \begin{tabular}{@{}cc@{}}
    \toprule
    Distanza [cm] & Rate [Hz]   \\ \midrule
    1.2      & $ 316.4\pm0.9 $ \\
    2.0      & $ 185.2 \pm  0.7$ \\
    3.0      & $ 107.8\pm 0.5$ \\
    4.0      & $ 70.1\pm $ 0.4\\
    5.0      & $ 49.1\pm0.3 $ \\
    6.0      & $ 36.2\pm 0.2$ \\ \bottomrule
    \end{tabular}%
    }
    \caption{Dati fit per verifica della legge dell'inverso del quadrato della distanza}
    \label{t:distanza}
    \end{table}


% %%%%%%%%%%%%%%%%%%%%%%%%%%%%%%%%%%%%%%%%%%%%%%%%%%%%%%%%%%%%%%%%%%%%%%
\section{Fit multipicco}\label{s:multipicco}





% %%%%%%%%%%%%%%%%%%%%%%%%%%%%%%%%%%%%%%%%%%%%%%%%%%%%%%%%%%%%%%%%%%%%%%
\section{Stima dell'efficienza relativa}\label{s:efficienza}


\end{document}
