%%% use twocolumn and 10pt options with the asme2ej format
\documentclass[twocolumn,10pt]{asme2ej}

\usepackage{epsfig} %% for loading postscript figures
\usepackage{subfigure}
\usepackage{titling}
\usepackage{siunitx}
\usepackage[hidelinks]{hyperref}
\usepackage{fancyhdr}
\pagestyle{fancy}
\fancyhead{}
\fancyfoot{}
\fancyfoot[R]{\thepage}

\def\figureautorefname{Fig.} 
\def\sectionautorefname{Sez.} 
\def\subsectionautorefname{Sez.} 
\def\subsubsectionautorefname{Sez.}


\pretitle{\begin{center}\linespread{1.2}\huge}
\posttitle{\par\end{center}\vspace{0.5em}}

\abovedisplayshortskip=0pt
\belowdisplayshortskip=0pt
\abovedisplayskip=-5pt
\belowdisplayskip=5pt


%% The class has several options
%  onecolumn/twocolumn - format for one or two columns per page
%  10pt/11pt/12pt - use 10, 11, or 12 point font
%  oneside/twoside - format for oneside/twosided printing
%  final/draft - format for final/draft copy
%  cleanfoot - take out copyright info in footer leave page number
%  cleanhead - take out the conference banner on the title page
%  titlepage/notitlepage - put in titlepage or leave out titlepage
%  
%% The default is oneside, onecolumn, 10pt, final

\date{}
\title{{\huge\bfseries Laboratorio di Fisica} - {\LARGE A.A. 2020/2021} \\ 
    {\LARGE Docenti: A. Garfagnini - M. Lunardon} \\ {\Huge\bfseries Fotodiodo}}


%%% first author
\author{Cerrone Vanessa
    \affiliation{
    1200361\\
    vanessa.cerrone@studenti.unipd.it
    }	
}

%%% second author
\author{Cigagna Simone
    \affiliation{
	1193992\\
    simone.cigagna@studenti.unipd.it
    }	
}

%%% third author
\author{Lai Nicolò
    \affiliation{
	1193976\\
    nicolo.lai@studenti.unipd.it
    }	
}


\begin{document}


\maketitle    


% %%%%%%%%%%%%%%%%%%%%%%%%%%%%%%%%%%%%%%%%%%%%%%%%%%%%%%%%%%%%%%%%%%%%%%
\section{Introduzione}\label{s:introduzione}
Si vuole analizzare lo spettro dei fotoni emessi dall'Americio-241 con un rivelatore al Silicio tipo PIN,
dotato di preamplificatore di carica. L’hardware, cioè i moduli di elettronica, sono stati pre-impostati
in condizioni standard, con shaping time pari a $3\mu\si{\second}$, in modo da ottimizzare il rapporto
segnale rumore. Preliminarmente, tramite il software di acquisizione, si registra uno spettro per 
identificare i picchi principali, a 60keV e 14-18keV. \\
Nella \autoref{s:attenuazione} si analizzerà il picco a 60keV in presenza di materiali di diverso spessore,
al fine di calcolare i relativi coefficienti di assorbimento. Nella  \autoref{s:distanza}, si effettueranno 
misure al variare della distanza della sorgente, per verificare che i dati seguano l'andamento atteso. 
Un'analisi dettagliata dello spettro verrà presentata nella sezione \autoref{s:multipicco}. 




% %%%%%%%%%%%%%%%%%%%%%%%%%%%%%%%%%%%%%%%%%%%%%%%%%%%%%%%%%%%%%%%%%%%%%%
\section{Coefficiente di assorbimento}\label{s:attenuazione}
Il coefficiente di assorbimento di un dato materiale



% %%%%%%%%%%%%%%%%%%%%%%%%%%%%%%%%%%%%%%%%%%%%%%%%%%%%%%%%%%%%%%%%%%%%%%
\section{Misure in funzione della distanza}\label{s:distanza}









% %%%%%%%%%%%%%%%%%%%%%%%%%%%%%%%%%%%%%%%%%%%%%%%%%%%%%%%%%%%%%%%%%%%%%%
\section{Fit multipicco}\label{s:multipicco}





% %%%%%%%%%%%%%%%%%%%%%%%%%%%%%%%%%%%%%%%%%%%%%%%%%%%%%%%%%%%%%%%%%%%%%%
\section{Stima dell'efficienza relativa}\label{s:efficienza}


\end{document}
